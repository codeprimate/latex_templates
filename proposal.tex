\documentclass{article}
\usepackage{multicol}
%\usepackage[left=1.5in,right=1in,top=1in,bottom=1in]{geometry}

% Use xelatex for font support
\usepackage{fontspec}
\setmainfont[Scale=1.2,AutoFakeBold=1.5,AutoFakeSlant=0.4]{Doves Type}
%%

% For graphics
\usepackage{graphics}
%%

% For footers
\usepackage{fancyhdr}
\usepackage{lastpage}
\pagestyle{fancy}
\fancyfoot[C]{\large{\thepage}/\small{\pageref{LastPage}}}
%%

% For tables
\usepackage{array}
\usepackage{booktabs}
%%

\title{Proposal Template}
\author{Your Name}
\date{\today} % Optional

\begin{document}

\maketitle

\section{Introduction}
\begin{description}
    \item[Problem Statement:] Clearly define the problem or opportunity that this feature aims to address.
    \item[Goal:] State the overarching goal of the feature proposal.
\end{description}

\section{Background}
\begin{description}
    \item[Context:] Provide context about the existing system, its users, and their needs.
    \item[Existing Solutions:] Briefly discuss any current solutions or workarounds.
\end{description}

\section{Feature Description}
\begin{description}
    \item[Feature Functionality:] Explain what the feature will do and how it addresses the problem.
    \item[User Experience:] Describe the intended user experience or interface changes.
\end{description}

\section{Implementation Plan}
\begin{description}
    \item[Feature Overview:]
    Provide a brief description of the feature to be implemented, including its purpose and expected benefits.

    \item[Technical Requirements:]
    Specify the technical prerequisites and dependencies required for developing this feature. This may include hardware, software, libraries, or frameworks.

    \item[Architecture Diagrams:] If applicable, include architectural diagrams to illustrate the system's structure with the new feature.

    \item[Development Environment Setup:]
    Outline the setup of the development environment, including tools, IDEs, version control systems, and any specific configurations necessary for the feature development.

    \item[Development Tasks:]
    List the specific development tasks required for the feature. This may include user interface design, backend development, database schema changes, and any external API integrations.

    \item[Code Development:]
    Detail the coding process, including the programming languages, best practices, and design patterns to be followed. Provide comprehensive code examples that are idiomatic and adhere to the organization's coding standards.
    
    \item[Code Examples:] Include idiomatic code examples, especially related to Ruby on Rails, to illustrate how the feature will be implemented.

    \item[Testing Strategy:]
    Describe the testing approach for the feature. This should encompass unit tests, integration tests, and user acceptance testing. Outline the criteria for success and the testing tools to be used.

    \item[Error Handling and Exception Handling:]
    Explain how error conditions and exceptions will be handled in the code, ensuring robustness and graceful failure. Provide code examples for common error scenarios.

    \item[Security Considerations:]
    Highlight any security-related aspects of the feature, such as data encryption, access control, and input validation. Address security best practices for this specific feature.

    \item[Performance Optimization:]
    Describe strategies for optimizing the performance of the feature, including database query optimization, caching, and load testing. Provide code examples related to performance improvements.

    \item[Documentation Updates:]
    Outline the necessary updates for documentation, including code comments, user guides, and API documentation. Ensure that all changes are well-documented.

    \item[Collaboration and Version Control:]
    Explain the collaboration tools and version control system to be used, as well as the branching strategy for code development.

    \item[Testing and QA Coordination:]
    Discuss how testing and quality assurance activities will be coordinated, including the involvement of QA teams and their testing schedules.

    \item[Deployment Plan:]
    Provide a plan for deploying the feature, including the target environment, deployment scripts, and rollback procedures. Address any specific considerations for staging and production environments.

    \item[Monitoring and Error Handling in Production:]
    Describe how the feature will be monitored in production, including error tracking, logging, and alerting mechanisms.

    \item[Rollout Strategy:]
    Explain the strategy for gradually rolling out the feature to users. Consider a phased or A/B testing approach, if applicable.

    \item[User Training and Support:]
    If the feature introduces changes for end-users, outline the training and support plan to help users adapt to the new functionality.

    \item[Post-Implementation Analysis:]
    Specify the metrics and KPIs to be monitored post-implementation to assess the feature's performance and user feedback.

    \item[Contingency Plan:]
    Address potential risks and contingencies if issues or unexpected challenges arise during implementation.

    \item[Timeline and Milestones:]
    Provide a timeline with milestones and deadlines for each phase of the implementation plan, helping to track progress and stay on schedule.

    \item[Responsibilities:]
    Define the roles and responsibilities of team members involved in the feature development and implementation process.

    \item[Dependencies:]
    Identify any dependencies on external factors, other projects, or teams that may impact the feature's development.

\end{description}

\section{Testing and Quality Assurance}
\begin{description}
    \item[Testing Plan:] Outline the testing strategy, including unit tests, integration tests, and user acceptance testing.
    \item[Error Handling:] Detail how potential error conditions will be handled, emphasizing your focus on production code quality.
\end{description}

\section{Impact Assessment}
\begin{description}
    \item[Potential Impacts:] Discuss the potential positive and negative impacts of the feature on the system, users, and organization.
    \item[Alignment with Goals:] Explain how the feature aligns with the organization's long-term goals and strategies.
\end{description}

\section{Prioritization}
\begin{description}
    \item[MoSCoW Method:] Prioritize the feature's importance using the Must have, Should have, Could have, Won't have method or other relevant criteria.
\end{description}

\section{Review and Approval}
\begin{description}
    \item[Stakeholder Review:] Mention the stakeholders and teams that should review and approve the proposal.
    \item[Sign-Off:] Describe the approval process and the expected sign-off.
\end{description}

\section{Implementation}
\begin{description}
    \item[Implementation Plan:] Outline how the feature will be developed, tested, and deployed.
\end{description}

\section{Deployment}
\begin{description}
    \item[Deployment Plan:] Explain the controlled deployment process, including monitoring and rollback plans.
\end{description}

\section{Documentation}
\begin{description}
    \item[User Documentation:] Describe the documentation needs for end-users.
    \item[Internal Documentation:] Specify the documentation for developers and maintainers.
\end{description}

\section{Post-Deployment Analysis}
\begin{description}
    \item[Performance Evaluation:] Discuss how the feature's performance will be assessed post-deployment.
    \item[User Feedback:] Consider how user feedback will be collected and analyzed.
    \item[Adjustments:] Mention how any necessary adjustments will be made based on post-deployment analysis.
\end{description}

\section{Continuous Improvement}
\begin{description}
    \item[Learning and Future Iterations:] Emphasize how insights from post-deployment analysis will inform future feature proposals and product development.
\end{description}

\section{Conclusion}
\begin{description}
    \item[Summary:] Recap the key points of the proposal.
    \item[Recommendation:] Provide a clear recommendation on whether the feature should be pursued.
\end{description}

% Vertical fill between pages
\vspace*{\fill}
\hrulefill
\vspace*{\fill}

\newpage 

\tableofcontents
\addcontentsline{toc}{section}{List of Tables}
\addcontentsline{toc}{section}{List of Figures}

\listoftables
\listoffigures

\end{document}
